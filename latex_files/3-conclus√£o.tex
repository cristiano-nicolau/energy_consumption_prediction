\section{Conclusion}

This project proved to be incredibly challenging, not only due to the complexity of the task but also because of the considerable effort required in working with the data. From the very beginning, we were confronted with difficult decisions, especially regarding the proper way to handle and structure the data. There were countless moments when we questioned whether we were on the right track, particularly when faced with the complexities of feature engineering and model selection.

One of the most challenging aspects of the project was deciding how to properly split the dataset—whether to divide the data by rows or by buildings. Initially, we opted for a traditional percentage-based split by rows into training, validation, and test sets. However, this raised an important question: would a more realistic and robust evaluation involve splitting by buildings, training the model on certain buildings and testing it on completely unseen ones? This doubt lingered throughout the project and often made us question whether our current approach truly reflected the generalization capability of our models.

Despite these hurdles, the project provided us with valuable insights into neural networks, particularly in understanding how to balance model performance and generalization. 

Ultimately, the project was a rewarding experience that allowed us to test our knowledge, learn new techniques, and refine our problem-solving skills. The difficulties we faced only strengthened our understanding of the intricacies involved in predictive modeling and provided us with a robust foundation for tackling similar challenges in the future. While the path was often uncertain and the road ahead unclear, the results we achieved gave us confidence in our ability to navigate complex problems with creativity and perseverance.

